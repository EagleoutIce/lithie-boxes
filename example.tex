\errorcontextlines 99999
\documentclass[twocolumn]{article}

\usepackage[T1]{fontenc}
\usepackage[utf8]{inputenc}

\usepackage[main=ngerman,english]{babel}

\usepackage{microtype,blindtext,XCharter,wrapfig,hyperref}
\usepackage[lockhigh]{lithie-boxes}

\SetBoxesPlain{bemerkung,definition*}

\hypersetup{pdfborder=1 1 0,colorlinks=true,allcolors=purple}

\LoadStyle{limerence}

\setlength{\columnsep}{1.55em}

\begin{document}
\section{Ich bin ein Abschnitt}
    % TODO: there has to be a better way!
\begin{figure*}
    \begin{definition}{Ich schwäbe, Mammi.}
        \vspace*{-\baselineskip}\begin{wrapfigure}[3]{r}{0.185\linewidth}
            \centering\vspace*{-2.33\baselineskip}\begin{tikzpicture}
                \draw circle [radius=1cm];
                \draw (225:1.25cm) -- (45:1.25cm);
            \end{tikzpicture}
        \end{wrapfigure}Ja und auf mir liegt der ganze Föküs. Den nutze ich jetzt auch! Zum Beispiel mit diesen sinnbefreiten Sinnessetzen zum besinnten Sinnentwandeln sinniger Sinnlichkeiten einer einsinnigen Ersinnung im Sinne der sinnvollen Anklage sinisterer Singles. 
    \end{definition}
\end{figure*}
\begin{figure*}
    \begin{bemerkung}{Eine freilich schwebende Bemerkung}
        \blindtext
    \end{bemerkung}
\end{figure*}

    Hier verweisen wir auf die Zukunft: \nameref{def:exdef}.

    \blindtext
    \begin{bemerkung*}{Nimm mich am besten gar nicht wahr\ldots}
        ich bin eine Randbemerkung. Aber eine wichtige!
    \end{bemerkung*}
    \blindtext[4]

    \begin{definition}{Super Titel}
        Ich bin die Sonne, die Wonne, der Glanz, da um mich tanzt; die Welt in der Hand und es trifft sich zum
        Schank; ach dann und wann ein jed-wed jeder der Lächeln noch kann.
    \end{definition}

    \blindtext[2]

    \begin{bemerkung}{Hat jemand die Eisenbahn gesehen?}
        Ich bin eine Bemerkung und eine Erinnerung an mich, dass ich etwas einbauen
        muss um eine Kurzform für einen Titel zu ermöglichen. Dies habe ich nun getan, aber ich würde dennoch genre einen Tanz beginnen ist das nicht fabelhaft?
    \end{bemerkung}

    \blindtext

    \begin{beispiel}{Ich bin ein noch viel bessers Beispiel}
        Super duper Beispiel Weispiel nicht war du bist leitspiel dreitspiel.
    \end{beispiel}

    \blindtext

    \begin{definition*}{Die Wichtig-Box}
        Hör doch! Hör doch! Hör doch die Melodie, die sich -- klangverliebt -- lieblich durch die Straßen zieht. 
        Seh doch! Seh doch! Sieh -- alle diese Bänder die sich dort
        als Farben ziehn; und so stehn' sie am Himmel und lebten
        schon im Glanz; als die Bänder noch verschlungen und
        die Noten noch im Tanz\ldots
    \end{definition*}

    \blindtext[1]

    \begin{satz}{Supersatz}
        Ja ich, mein kleiner Spatz, ich bin voll der Super-Satz.
    \end{satz}

    Isch bin en füller denn mir brauchet hier noch oin zwoi Zeilööön.  Und die erschaffe ich damit glaube ich ganz effitschient wie mir san. Also\ldots{} guck mr mal wies ausschaut nunele net woar? Wird bestimmt dr puuuure hammer\ldots
    Schade mittlerweile bin i outdated.
    \begin{definition}[Kurzer Ersatz für lang]{Ich bin der lange Titel der, dich frei und unbeschwert, durch die Welt sich selbst erfährt und heimlich seine Tagen gährt.}
        Man kennt mich als die Dunkelheit, die Petroleum den Flammen zeigt, und in die der Rauch aufsteigt.
        Für mich schlägt die Mitternacht derer Luft die Welt
        vermacht.
    \end{definition}

    \blindtext

    \begin{beispiel*}{Dies ist ein beispiel}
        \blindtext
    \end{beispiel*}

    \blindtext

    \begin{definition*}{Lange Titel mögen wir; wenn auch die Box an sich nicht so. Drum brechen wir hier ab}
        \label{def:exdef}Ach sie lächeln, strahlen, freuen, sich an all den Leuten,
        die noch heute -- vielleicht morgen -- sich auch um den Andern sorgen. Und von Dauer lebt nicht lange schon der Zeiten wegen
        der, welcher bei geöffnetem Fenster sein Papier nicht beschwert.
    \end{definition*}

    \blindtext

    % TODO: show all lists?
    \listofdefinitions
    \listofremarks
    \listoftheorems
    \listofexamples
\end{document}